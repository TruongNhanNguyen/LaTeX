\begin{minipage}[t]{0.55\textwidth}
    \begin{center}
        \large ĐẠI HỌC QUỐC GIA THÀNH PHỐ HỒ CHÍ MINH \\
               TRƯỜNG ĐẠI HỌC CÔNG NGHỆ THÔNG TIN\\[0.75cm]
        \textbf{\underline{BỘ MÔN TOÁN LÝ}}
    \end{center}
\end{minipage}
\hfill
\begin{minipage}[t]{0.5\textwidth}
    \begin{center}
        \textbf{ĐỀ THI GIỮA KỲ MÔN ĐẠI SỐ TUYẾN TÍNH }\\
        Học kỳ I - Năm học 2020-2021\\
        \vspace{1cm}
        Ngày thi: 16/12/2020\\
        Thời gian làm bài: \textbf{60 phút}
    \end{center}
\end{minipage}
 
 \vspace{2cm}
 
 \recprob{1}{3}
 Cho các ma trận thực: $A=\begin{pmatrix} 
                           \phantom{-} 1 & -1 & \phantom{-} 2 \\
                            -1 & \phantom{-}2 &\phantom{-} 1 \\
                           \phantom{-} 2 & -3 &\phantom{-} 2
                        \end{pmatrix}$,
                        $B=\begin{pmatrix}
                           \phantom{-}5 & -8 & \phantom{-} 4 \\
                           -6 & \phantom{-}7 & \phantom{-}4 \\
                           \phantom{-}2 & \phantom{-}1 & -9
                        \end{pmatrix}$

\begin{itemize}
    \item [\textbf{a)}] Tìm các ma trận $A^TB-5(B^TA)^2$ và $A^3-2BA+9A-B$.
    \item[\textbf{b)}]  Tìm ma trận vuông $X$ thỏa mãn $XA=B^T$.
\end{itemize}
 
 \recprob{2}{3}
 Hãy giải và biện luận hệ phương trình tuyến tính sau, trên trường số thực:
 
 $$\left\{\begin{array}{lrr}
    3x_3 + x_1 +mx_2 & = 2 & \\
    3x_2 + mx_3 +2x_1 & = 3 & \text{ với $m$ là tham số thực.}\\
    x_2 - x_3 +x_1 & =1 & 
 \end{array}
 \right.$$ 
 
 \recprob{3}{2}
  Trên $M_2(\mathbb{R})$ là không gian các ma trận vuông, thực, cấp 2, cho tập hợp:
  
  $$W = \left\{\left.A=\begin{pmatrix}
                      5a+c & 4b \\
                      b-2c & a+3b
                 \end{pmatrix}\right|a,b,c\in\mathbb{R}\right\}$$.
  
  Hỏi $W$ có phải là không gian vector con của $M_2(\mathbb{R})$ hay không? Vì sao?\\
 
 \recprob{4}{1.5}
 Trên $\mathbb{R}^3$ cho tập hợp $S=\left\{\alpha_1=(1,-3,2), \alpha_2=(-1,4,0), \alpha_3=(-2,4,m^2+4)\right\}.$
 Tìm điều kiện của $m$ để $S$ là phụ thuộc tuyến tính.\\
 
 \recprob{5}{0.5}
 Một công ty sản xuất linh kiện máy tính cần dùng 3 loại nguyên liệu (ký hiệu là :$S_1, S_2, S_3$) với thành phần tỷ lệ các nguyên liệu (tính bằng \%) như sau:
 
 $$\begin{matrix}
        & \ce{Al} & \ce{Si} & \ce{Fe} \\
    S_1 &    2    &    1    &    97   \\
    S_2 &    3    &    4    &    93   \\
    S_3 &    1    &    2    &    97   
    \end{matrix}$$
    
Hỏi công ty cần phối trộn 3 loại nguyên liệu với tỷ lệ như thế nào để có thể sản xuất một linh kiện máy tính có tỷ lệ hỗn hợp các nguyên liệu (tính bằng \%) là:
  
$$\ce{Al} = 1,6 \%, \ce{Si} = 1,8 \%, \ce{Fe} = 96,6 \%$$
 
\begin{center}
    \vspace{2cm}
    \rule{10cm}{0.5pt}
    
    \textbf{Hết}\\
    \textit{Cán bộ coi thi không giải thích gì thêm}\\
    \textit{Thí sinh không được sử dụng tài liệu.}    
\end{center}
