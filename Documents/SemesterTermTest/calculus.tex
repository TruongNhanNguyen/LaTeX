\begin{minipage}[t]{0.55\textwidth}
    \begin{center}
        \large ĐẠI HỌC QUỐC GIA THÀNH PHỐ HỒ CHÍ MINH \\
               TRƯỜNG ĐẠI HỌC CÔNG NGHỆ THÔNG TIN\\[0.75cm]
        \textbf{\underline{BỘ MÔN TOÁN LÝ}}
    \end{center}
\end{minipage}
\hfill
\begin{minipage}[t]{0.4\textwidth}
    \begin{center}
        \textbf{ĐỀ THI GIỮA KỲ MÔN GIẢI TÍCH}\\
        Học kỳ I - Năm học 2020-2021\\
        \vspace{1cm}
        Ngày thi: 15/12/2020\\
        Thời gian làm bài: \textbf{60 phút}
    \end{center}
\end{minipage}
 
 \vspace{2cm}

\recprob{1}{2}
Chứng minh giới hạn sau không tồn tại: $$\lim\limits_{(x,y)\to(2,0)} \dfrac{(x-2)y^2}{(x-2)^3+y^4} $$

\recprob{2}{2}
Tìm cực trị hàm hai biến sau: $$f(x,y)=xy+\dfrac{1}{x}+\dfrac{1}{y}$$

\recprob{3}{3}
   \begin{itemize}
       \item[\textbf{a)}] Khảo sát sự hội tụ của chuỗi số sau: $\displaystyle\sum\limits_{n=1}^{+\infty}\dfrac{1}{3^n}\left(\dfrac{n+3}{n+1}\right)^{n^2} $
       \item[\textbf{b)}]  Tìm miền hội tụ của chuỗi số:
       $\displaystyle\sum\limits_{n=1}^{+\infty}\dfrac{(x-1)^n}{\sqrt{n^2+2}\cdot3^n} $
   \end{itemize}

\recprob{4}{3}
Xét sự hội tụ của các tích phân sau:

\begin{itemize}
    \item[\textbf{a)}] $I_1=\displaystyle\int\limits_{0}^{2}\dfrac{\sin{(2-x)\mathrm{d}x}}{\sqrt[3]{8-x^3}\cdot(e^{4-x^2}-1)}$
    \item[\textbf{b)}]  $I_2=\displaystyle\int\limits_{1}^{+\infty}\dfrac{(x+2)\mathrm{d}x}{(2x^3+1)\ln{x}} $
\end{itemize}


\begin{center}
    \vspace{4cm}
    \rule{10cm}{0.5pt}
    
    \textbf{Hết}\\
    \textit{Cán bộ coi thi không giải thích gì thêm}\\
    \textit{Thí sinh không được sử dụng tài liệu.}    
\end{center}
